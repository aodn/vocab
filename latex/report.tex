%% latex x.tex ; evince x.dvi

\documentclass[10pt,a4paper]{article}
% \documentclass[12pt]{report}
\usepackage{amsmath}
\usepackage{hyperref}

% indent the first para after a new section
\usepackage{indentfirst}


% italic environment
\newenvironment{italicquotes}
{\begin{quote}\itshape}
{\end{quote}}

\title{Whoot!}
\date{}
\begin{document}

  \maketitle
    \begin{flushleft}
  \setlength{\parindent}{5ex}



\section{High level requirements}

\begin{italicquotes} hopefully italic  

  and more italic
  \end{italicquotes} 


\subsection{Resolvable content should be structured (or at least be able to be queried)
  using an RDF/SKOS encoding model. Content may however be adorned by other
  languages/metadata models (e.g. RDFS, OWL, Dublin Core).* }

  SISSVoc API - is orientated around Concept, Collection and .

  It's assumed that being decorated with other RDF content.
  

	SISSVoc provides a linked data API for publishing SKOS vocabularies.
  SKOS provides a standard vocabulary for thesarui, classifications, taxonomies
  and controlled vocabularies using RDF.
   According to Cox, SKOS lacks the expresitivity of languages such as OWL.
  However common metadata standards like Dublin Core are supported.
    It is anticipated that SKOS can be decorated with any data/metadata that can
  be encoded in RDF triples.


	Additionally the SKOS Extensions RDF Vocabulary are a set of terms extending
	the SKOS Core vocabulary to support some common features of knowledge
	organisation systems, especially thesauri. (link) The extensions would appear
	to be designed to cover common needs for representing authoring and publishing.

	Resolvable content - can be extracted SKOS Concepts from the already developed
	relational vocab database demonstrates another approach which can then be
	ingested to a RDF persistence. An example script has been created to demonstate
	the feasibility of this approach. Alternately a SPARQL interface to RDF could
	be used over the db, to form RDF/SKOS content.


\subsection{} 
\begin{italicquotes} 
3. It should be possible to access vocabularies and their terms via an 
(administratively) customisable Web-client interface and service interfaces.*

\end{italicquotes} 

SiSSVoc does not offer support for creation and update of vocabulary terms - 
since it is implemented on top of a read-only SPARQL endpoint/interface.  
According to the ANDS handback… has extended SISSVOC, with a web supporting 
the editing SKOS files at file level - presumably by interacting with the 
file-store. Some support for web-based versioning and author management is 
available. There is a potential for conflict with the \texttt{control\_vocab\_db}. And a 
question. if we maintain current db.
 
SiSSVoc running as a local instance supports customisation, image branding 
changes can be achieved and examples are given using simple css configuration. 
Examples are given in the source example.


SISSVOC advice using a an external vocabulary content maintained using rdf 
editors (such as Protege or TopBraid Composer ). which ensure consistency of 
relationships between resources is maintained. And then generate the rdf files. 

For web-based - maintenance environment - Enterprise Vocabulary NET and 
PoolParty Thesaurus Server. 




\subsection{} 
\begin{italicquotes} 
  Resolvable content should be structured (or at least be able to be queried) 
  using an RDF/SKOS encoding model. Content may however be adorned by other 
  languages/metadata models (e.g. RDFS, OWL, Dublin Core).* 
\end{italicquotes} 
  

  SISSVoc provides a linked data API for publishing SKOS vocabularies. SKOS 
  provides a standard vocabulary for thesarui, classifications, taxonomies and 
  controlled vocabularies using RDF. According to Cox, SKOS lacks the 
  expresitivity of languages such as OWL. However common metadata standards like 
  Dublin Core are supported. It is anticipated that SKOS can be decorated with 
  any data/metadata that can be encoded in RDF triples. 

  Additionally the SKOS Extensions RDF Vocabulary are a set of terms extending 
  the SKOS Core vocabulary to support some common features of knowledge 
  organisation systems, especially thesauri. (link) The extensions would appear 
  to be designed to cover common needs for representing authoring and publishing .

  Resolvable content - can be extracted SKOS Concepts from the already developed 
  relational vocab database demonstrates another approach which can then be 
  ingested to a RDF persistence. An example script has been created to demonstate 
  the feasibility of this approach. Alternately a SPARQL interface to RDF could 
  be used over the db, to form RDF/SKOS content.




\subsection{y}
  more crap


\begin{subsection}{Resolvable content should be structured (or at least be able to be queried)
  using an RDF/SKOS encoding model. Content may however be adorned by other
  languages/metadata models (e.g. RDFS, OWL, Dublin Core).  }

	SISSVoc provides a linked data API for publishing SKOS vocabularies.


  \begin{textit}{
    should be italic
  }
  \end{textit}


\end{subsection}





  \end{flushleft}


  \textit{should be italic}


\LaTeX{} is a document preparation system for the \TeX{}
  typesetting program. It offers programmable desktop
  publishing features and extensive facilities for
  automating most aspects of typesetting and desktop
  publishing, including numbering and cross-referencing,
  tables and figures, page layout, bibliographies, and
  much more. \LaTeX{} was originally written in 1984 by
  Leslie Lamport and has become the dominant method for
  using \TeX; few people write in plain \TeX{} anymore.
  The current version is \LaTeXe.

  \begin{flushright}
  a para
  \textbf{should be bold}
  \textit{should be italic}

  \end{flushright}



  \begin{flushleft}
  My footnote\footnote{An example footnote.}
  Another footnote\footnote{Another footnote.}

  \hyperref[cnn]{http://www.cnn.com}

  \url{http://www.google.com}
  \href{http://www.google.com}{my google}
  \href{http://www.cnn.com}{http://www.cnn.com}

  % http://www.google.com
  \end{flushleft}
  more stuff
  more stuff2

  % This is a comment, not shown in final output.
  % The following shows typesetting power of LaTeX:
  \begin{align}
    E_0 &= mc^2                              \\
    E &= \frac{mc^2}{\sqrt{1-\frac{v^2}{c^2}}}
  \end{align}
\end{document}

