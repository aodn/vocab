%% sudo apt-get install texlive-latex-extra
%% latex x.tex ; evince x.dvi

\documentclass[10pt,a4paper]{article}
% \documentclass[12pt]{report}
\usepackage{amsmath}

% gives us \url 
\usepackage{hyperref}

% indent the first para after a new section
\usepackage{indentfirst}

% missing - \usepackage{sectsty}


% control section fonts - it does work
\usepackage{sectsty}
\subsectionfont{\fontsize{10}{10}\selectfont}

% italic environment
\newenvironment{italicquotes}
{\begin{quote}\itshape}
{\end{quote}}

% bullet point lists
\let\Item\item
\newcommand\SpecialItem{\renewcommand\item[1][]{\Item[\textbullet~\bfseries##1]}}
\renewcommand\enddescription{\endlist\global\let\item\Item}

\title{Whoot!}
\date{}
\begin{document}
\SpecialItem

  \maketitle
    \begin{flushleft}
  \setlength{\parindent}{5ex}

% 
% \section{High level requirements}
%   \item[Text] more text
%     and below
%   \item[and some] more text
%   \item[] empty
% \begin{italicquotes} hopefully italic  
%   and more italic
%   \end{italicquotes} 
% \url{http://www.uni.edu/~myname/best-website-ever.html}
% 

% need to be consisistent.

% OK
% 1 
  \subsection{
   Main function is to provide a resolvable endpoint for a vocabulary and its
  included terms (and details) using persistent identifiers.* 
  }

  For each item in the vocabulary, persistent identifiers {URI} should
  resolve to a description of the item.  This is necessary for direct reference to
  vocabulary items, and in-line links within datasets.

  \item[SISSVoc] as a standalone application has no inherant support for managing persistent
  identifiers. The SISSVOC paper describes a deployment \footnote{ See Figure 3,
  SISSVoc: A Linked Data API for SKOS vocabularies.

  \url{http://www.semantic-web-journal.net/system/files/swj658.pdf} }, using a
  Persistent Identifier Service is used to map persistent URI resources to SISSVoc
  webservice urls but doesn't give futher details on the implementation or
  whether and external provider was chosen. 

  \item[ANDS] controlled vocabulary services build upons SISSVOC, although it appears
  their persistent identifier service is a more general capability. According to
  documentation, ANDS has a, 

  \begin{italicquotes} [...] simple HTTP-based interface, [which] ensures that
  identifier services can be integrated easily into existing data management
  workflows.  \end{italicquotes}
  According to their documentation, ANDS does undertake to persist the infrastructure for required for
  keeping its identifiers online. Ideally the service would be integrated
  with other SISSVoc management functionality, although this was not tested.

  \item[]Another alternative, would be to use a service such as \url{purl.org}  \footnote{ The most
  prominent instances of such schemes are PURLexternal link, which has been used
  by the National Library of Australia, and ARKexternal link, at the California
  Digital Library.  }. 

  \item[]If desired it would probably also be simple to publish identifiers under using an
  AODN or EMII dns based url. In this context, seegrid appear to have developed their own
  PID service used in conjunction with SISSOV  \footnote{
  https://www.seegrid.csiro.au/wiki/Siss/PIDService}  
  
  \item[]It's uncertain what capabilities LDP has for persisent identifiers.


% also supports versioning and management of PID could
%% ok, we got this restful stuff confused. GET,POST,DELETE,UPDATE... etc.
%% are html commands.

%   \# SISSVoc Deployment complemented by PID service
%   PURL. SISSVOC used a per
%   \url{https://www.seegrid.csiro.au/wiki/Siss/PIDService} 
% 
%   
%   SiSSVoc - locally or externally hosted (ANDS) supports using persistent
%   identifiers.  Article tals branding. 4 things - expected.  Talk about rdf.
% 
%   SiSSVOC is built upon rdf, and is a restful publishing api. 
%%   The DNS resolvable component of the uri, it is expected that ANDs will maintain
% 
%   ANDS and IMOS has extensive experience in managing in chef the DNS component of
%   names. Alternatively ANDS
% 


%% OK
% 2
\subsection{Resolvable content should be structured (or at least be able to be queried)
  using an RDF/SKOS encoding model. Content may however be adorned by other
  languages/metadata models (e.g. RDFS, OWL, Dublin Core).* }

  SISSVoc provides a linked data API for publishing SKOS vocabularies.  SKOS
provides a standard vocabulary for thesarui, classifications, taxonomies and
controlled vocabularies using RDF.

  The SISSVoc web-service API supports URI patterns aligned with the SKOS
vocabulary model. This includes patterns for SKOS Concept,
ConceptScheme and Collection. Further URI patterns are provided to discover
broader and narrower terms in transitive and non-transitive forms and according
to text based labels. 

  SKOS content can be decorated with other RDF based
content and persisted to the underlying triple store. However SISSVOC provides
no direct support URI patterns for search and discovery of such content. As an
alternative, a sparql interface would provide a
such query/search mechanism for non-SKOS content such as RDFS and OWL classes. 


%    According to Cox, SKOS lacks the expresitivity of languages such as OWL.
%   However common metadata standards like Dublin Core are supported.
%     It is anticipated that SKOS can be decorated with any data/metadata that can
%   be encoded in RDF triples.
% 
%   Additionally the SKOS Extensions RDF Vocabulary are a set of terms extending
%   the SKOS Core vocabulary to support some common features of knowledge
%   organisation systems, especially thesauri. (link) The extensions would appear
%   to be designed to cover common needs for representing authoring and publishing.
% 
%   Resolvable content - can be extracted SKOS Concepts from the already developed
%   relational vocab database demonstrates another approach which can then be
%   ingested to a RDF persistence. An example script has been created to demonstate
%   the feasibility of this approach. Alternately a SPARQL interface to RDF could
%   be used over the , to form RDF/SKOS content.
% 

% 3 
  \subsection{} 
  \begin{italicquotes} 
  It should be possible to access vocabularies and their terms via an 
  (administratively) customisable Web-client interface and service interfaces.*

  \end{italicquotes} 

  - still waiting on ANDS to give a test account.

  - need to mention the handbook.

  SiSSVoc as a standalone application does not support for creation and update
  of vocabulary terms.  - since it is implemented on top of a read-only SPARQL
  endpoint/interface.  According to the ANDS handback… has extended SISSVOC, with
  a web supporting the editing SKOS files at file level - presumably by
  interacting with the file-store. Some support for web-based versioning and
  author management is available. There is a potential for conflict with the
  \texttt{control\_vocab\_db}. And a question. if we maintain current db.

  SiSSVoc running as a local instance supports customisation, image branding
changes can be achieved and examples are given using simple css configuration.
Examples are given in the source example.

  SISSVOC advice using a an external vocabulary content maintained using rdf
editors (such as Protege or TopBraid Composer ). which ensure consistency of
relationships between resources is maintained. And then generate the rdf files. 

  For web-based - maintenance environment - Enterprise Vocabulary NET and
PoolParty Thesaurus Server. 




\subsection{y}
  more crap


\begin{subsection}{Resolvable content should be structured (or at least be able to be queried)
  using an RDF/SKOS encoding model. Content may however be adorned by other
  languages/metadata models (e.g. RDFS, OWL, Dublin Core).  }

  SISSVoc provides a linked data API for publishing SKOS vocabularies.


  \begin{textit}{
    should be italic
  }
  \end{textit}


\end{subsection}





  \end{flushleft}


  \textit{should be italic}


\LaTeX{} is a document preparation system for the \TeX{}
  typesetting program. It offers programmable desktop
  publishing features and extensive facilities for
  automating most aspects of typesetting and desktop
  publishing, including numbering and cross-referencing,
  tables and figures, page layout, bibliographies, and
  much more. \LaTeX{} was originally written in 1984 by
  Leslie Lamport and has become the dominant method for
  using \TeX; few people write in plain \TeX{} anymore.
  The current version is \LaTeXe.

  \begin{flushright}
  a para
  \textbf{should be bold}
  \textit{should be italic}

  \end{flushright}



  \begin{flushleft}
  My footnote\footnote{An example footnote.}
  Another footnote\footnote{Another footnote.}

  \hyperref[cnn]{http://www.cnn.com}

  \url{http://www.google.com}
  \href{http://www.google.com}{my google}
  \href{http://www.cnn.com}{http://www.cnn.com}

  % http://www.google.com
  \end{flushleft}
  more stuff
  more stuff2

  % This is a comment, not shown in final output.
  % The following shows typesetting power of LaTeX:
  \begin{align}
    E_0 &= mc^2                              \\
    E &= \frac{mc^2}{\sqrt{1-\frac{v^2}{c^2}}}
  \end{align}
\end{document}

