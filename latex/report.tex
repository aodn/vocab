%% rm report.pdf ; pdflatex -halt-on-error report.tex  && evince report.pdf
%% latex x.tex ; evince x.dvi

\documentclass[10pt,a4paper]{article}
% \documentclass[12pt]{report}
\usepackage{amsmath}

% gives us \url 
\usepackage{hyperref}

% indent the first para after a new section
\usepackage{indentfirst}

% missing - \usepackage{sectsty}


% control section fonts - it does work
\usepackage{sectsty}
\subsectionfont{\fontsize{10}{10}\selectfont}

% italic environment
\newenvironment{italicquotes}
{\begin{quote}\itshape}
{\end{quote}}

% bullet point lists
\let\Item\item
\newcommand\SpecialItem{\renewcommand\item[1][]{\Item[\textbullet~\bfseries##1]}}
\renewcommand\enddescription{\endlist\global\let\item\Item}

\title{Whoot!}
\date{}
\begin{document}
\SpecialItem

  \maketitle
    \begin{flushleft}
  \setlength{\parindent}{5ex}

% 
% \section{High level requirements}
%   \item[Text] more text
%     and below
%   \item[and some] more text
%   \item[] empty
% \begin{italicquotes} hopefully italic  
%   and more italic
%   \end{italicquotes} 
% \url{http://www.uni.edu/~myname/best-website-ever.html}
% 

% need to be consisistent.






% OK
% 1 
  \subsection{
   Main function is to provide a resolvable endpoint for a vocabulary and its
  included terms (and details) using persistent identifiers.* 
  }

  For each item in the vocabulary, persistent identifiers {URI} should
  resolve to a description of the item.  This is necessary for direct reference to
  vocabulary items, and in-line links within datasets.

  \item[SISSVoc] as a standalone application has no inherant support for managing persistent
  identifiers. The SISSVOC paper describes a deployment \footnote{ See Figure 3,
  SISSVoc: A Linked Data API for SKOS vocabularies.

  \url{http://www.semantic-web-journal.net/system/files/swj658.pdf} }, using a
  Persistent Identifier Service is used to map persistent URI resources to SISSVoc
  webservice urls but doesn't give futher details on the implementation or
  whether and external provider was chosen. 

  \item[ANDS] controlled vocabulary services build upons SISSVOC, although it appears
  their persistent identifier service is a more general capability. According to
  documentation, ANDS has a, 

  \begin{italicquotes} [...] simple HTTP-based interface, [which] ensures that
  identifier services can be integrated easily into existing data management
  workflows.  \end{italicquotes}
  According to their documentation, ANDS does undertake to persist the infrastructure for required for
  keeping its identifiers online. Ideally the service would be integrated
  with other SISSVoc management functionality, although this was not tested.

  \item[]Another alternative, would be to use a service such as \url{purl.org}  \footnote{ The most
  prominent instances of such schemes are PURLexternal link, which has been used
  by the National Library of Australia, and ARKexternal link, at the California
  Digital Library.  }. 

  \item[]If desired it would probably also be simple to publish identifiers under using an
  AODN or EMII dns based url. In this context, seegrid appear to have developed their own
  PID service used in conjunction with SISSOV  \footnote{
  https://www.seegrid.csiro.au/wiki/Siss/PIDService}  
  
  \item[]LDP is a resource registry management system, however it's unknown 
  if this includes direct support for persisent identifiers.
    \footnote{ See 
    https://github.com/UKGovLD/ukl-registry-poc/wiki/Principles-and-concepts
  }


% LDP -   The API should, where reasonable, follow REST principles. Specifically that
% any resource in the system should be identified by a URI and be manipulable by
% standard HTTP verbs (GET, PUT, DELETE, PATCH).
% 

% also supports versioning and management of PID could
%% ok, we got this restful stuff confused. GET,POST,DELETE,UPDATE... etc.
%% are html commands.

%   \# SISSVoc Deployment complemented by PID service
%   PURL. SISSVOC used a per
%   \url{https://www.seegrid.csiro.au/wiki/Siss/PIDService} 
% 
%   
%   SiSSVoc - locally or externally hosted (ANDS) supports using persistent
%   identifiers.  Article tals branding. 4 things - expected.  Talk about rdf.
% 
%   SiSSVOC is built upon rdf, and is a restful publishing api. 
%%   The DNS resolvable component of the uri, it is expected that ANDs will maintain
% 
%   ANDS and IMOS has extensive experience in managing in chef the DNS component of
%   names. Alternatively ANDS
% 


%% OK
% 2
\subsection{Resolvable content should be structured (or at least be able to be queried)
  using an RDF/SKOS encoding model. Content may however be adorned by other
  languages/metadata models (e.g. RDFS, OWL, Dublin Core).* }

  SISSVoc provides a linked data API for publishing SKOS vocabularies.  SKOS
provides a standard vocabulary for thesarui, classifications, taxonomies and
controlled vocabularies using RDF.

  The SISSVoc web-service API supports URI patterns aligned with the SKOS
vocabulary model. This includes patterns for SKOS Concept,
ConceptScheme and Collection. Further URI patterns are provided to discover
broader and narrower terms in transitive and non-transitive forms and according
to text based labels. 

  SKOS content can be decorated with other RDF based
content and persisted to the underlying triple store independently of SISSVOC
functionality. SISSVOC provides
no direct support URI patterns for search and discovery of such content. As an
alternative, a sparql interface would provide a
such query/search mechanism for non-SKOS content such as RDFS and OWL classes. 

It is believed that LDP has no specific API support for SKOS resources.

%    According to Cox, SKOS lacks the expresitivity of languages such as OWL.
%   However common metadata standards like Dublin Core are supported.
%     It is anticipated that SKOS can be decorated with any data/metadata that can
%   be encoded in RDF triples.
% 
%   Additionally the SKOS Extensions RDF Vocabulary are a set of terms extending
%   the SKOS Core vocabulary to support some common features of knowledge
%   organisation systems, especially thesauri. (link) The extensions would appear
%   to be designed to cover common needs for representing authoring and publishing.
% 
%   Resolvable content - can be extracted SKOS Concepts from the already developed
%   relational vocab database demonstrates another approach which can then be
%   ingested to a RDF persistence. An example script has been created to demonstate
%   the feasibility of this approach. Alternately a SPARQL interface to RDF could
%   be used over the , to form RDF/SKOS content.
% 

% 3 
% NOT QUITE RIGHT - adminstratively customisable.
  \subsection{
  It should be possible to access vocabularies and their terms via an 
  (administratively) customisable Web-client interface and service interfaces.* } 


  SISSVoc provides a capable Web-client interface for read-only access vocabularies and their terms.

  In contrast to search, naviation and discovery, SiSSVoc as a standalone application has no direct 
  support for the creation and update of vocabulary terms. This is inherent to SISSVOC design, as a 
  lightweight web-api implemented over a read-only SPARQL endpoint/interface.  

%   the SISSVOC client interface supports customisation, image branding
%   which can be achieved via changes to css and js configuration.
%   Examples are given in the source code.
%    \footnote {
%       https://github.com/jyucsiro/sissvoc-runner/tree/master/sissvoc/resources
%     }
%    
  According to the ANDS handbook, ANDS provides web-based GUI support for editing SKOS files at file level. 
  \footnote { See 3.4.3, Editing Vocabularies \url{ http://www.ands.org.au/support/vocab-help-guide.pdf } 
  }

  Some support for web-based versioning and author management at the file level
is also available. Permissions to make change are tied to authority roles.

  \footnote {
    There is a need to consider whether file level versioning and management is
  sufficiently fine-grained. Also interaction with existing registry management
  already used in IMOS \texttt{control\_vocab\_db}. 
  }
  
% this doesn't quite belong, 
  For complex vocabulary authoring needs, SISSVOC authors suggest using an
external vocabulary editor to maintain content that can also ensure that consistency of
relationships between resources is maintained (eg. for OWL). 

  \begin{italicquotes} 
  Vocabulary content may be maintained using RDF editors (such as Protégé 4 or
TopBraid Composer 5 ), which ensure consistency of
relationships between resources is maintained, and
then generate RDF documents to transfer vocabulary
content from the maintenance to publication envi-
ronment, as outlined above. If a web-based vocabu-
lary maintenance environment is required, then tools
like TopQuadrant’s Enterprise Vocabulary Net 6 , and
the PoolParty Thesaurus Server 7 are available.
  \end{italicquotes} 

  \footnote {
    3.2. HTTP operations and REST behaviour, 
\url{http://www.semantic-web-journal.net/system/files/swj658.pdf} 
  }


% this REPEATS/CONFLICTS with subsection 3
% 4
\subsection{ 
  4. Most users require read only access to content.* 
}
  SISSVoc provides a capable Web-client interface for read-only access vocabularies and their terms.

  In contrast to search, navigation and discovery, SiSSVoc as a standalone application has no direct 
  support for the creation and update of vocabulary terms. This is inherent to SISSVOC design, as a 
  lightweight web-api implemented over a read-only SPARQL endpoint/interface.  

  ANDs extends SISSVoc cabpability by including support for uploading, and raw editing
  at file-level. 

  LDP provides a Web-client for fine grained registry management of term
resources. But note, this is a general capability lacking SKOS specific functionality.



\subsection{ 
  5. There should ideally be a service interface that is REST-based* and a SPARQL service end-point.
}

  SISSVoc is designed with a HTTP-based interface aligned with REST-based web
services. The URI patterns facilitate SKOS discovery and access. Note however,
that SISSVoc is not a full RESTful API, as it does not support HTTP operations
for update and deletion of resources.
  
  LDP is designed with a view to RESTFUL management of registry resources.
\footnote { https://github.com/UKGovLD/ukl-registry-poc/wiki/Api } It should be
noted that references to SKOS in the discussion of the LDP api apply to
registrars, not the SKOS contents of any particular registrars.  


  SPARQL (Simple Protocol and RDF Query Language) is an RDF query language able
to preserve, retrieve and manipulate data in RDF format and is a W3c standard.
SPARQL abstracts encoding model (XML, … etc), and is adapted to the
non-relational / graph structure of RDF.

  The SISSVOC RESTFUL URI patterns correspond very closely with SPARQL queries
which perform the heavy-lifting.  SISSVOC uses a SPARQL endpoint as the
mechanism to access RDF providing a strong separation of concerns.

  It could be possible to map the current \texttt{control\_vocab\_db} to expose a sparql
interface using a tool such as r2d.  The sparql interface would the provide the
endpoint SISSVOC and replacing the need for an additional persistence layer and
the need to ingest raw skos files.

  It is anticipated that an external SISSVOC provider such as ANDs would be
unlikely to expose a sparql interfaces.
   


\subsection{y}






  \end{flushleft}


  \textit{should be italic}


\LaTeX{} is a document preparation system for the \TeX{}
  typesetting program. It offers programmable desktop
  publishing features and extensive facilities for
  automating most aspects of typesetting and desktop
  publishing, including numbering and cross-referencing,
  tables and figures, page layout, bibliographies, and
  much more. \LaTeX{} was originally written in 1984 by
  Leslie Lamport and has become the dominant method for
  using \TeX; few people write in plain \TeX{} anymore.
  The current version is \LaTeXe.

  \begin{flushright}
  a para
  \textbf{should be bold}
  \textit{should be italic}

  \end{flushright}



  \begin{flushleft}
  My footnote\footnote{An example footnote.}
  Another footnote\footnote{Another footnote.}

  \hyperref[cnn]{http://www.cnn.com}

  \url{http://www.google.com}
  \href{http://www.google.com}{my google}
  \href{http://www.cnn.com}{http://www.cnn.com}

  % http://www.google.com
  \end{flushleft}
  more stuff
  more stuff2

  % This is a comment, not shown in final output.
  % The following shows typesetting power of LaTeX:
  \begin{align}
    E_0 &= mc^2                              \\
    E &= \frac{mc^2}{\sqrt{1-\frac{v^2}{c^2}}}
  \end{align}
\end{document}

